\documentclass[a4paper]{article}

\usepackage[english]{babel}
\usepackage[utf8x]{inputenc}
\usepackage{amsmath}
\usepackage{graphicx}
\usepackage[colorinlistoftodos]{todonotes}
\usepackage{array}

\title{Analyzing and categorizing Wikipedia articles using Python3 and TextBlob}
\author{
  Berenji, Sarah\\
  \texttt{sarah.berenji@gmail.com}
  \and
  Forstén, Andreas\\
  \texttt{andreasforsten@gmail.com}
  \and
  Leskelä, Hannes\\
  \texttt{hleskela@kth.se}
  \and
  Letzner, Josefine\\
    \texttt{joletzner@gmail.com}
}
\date{2015-10-04}
\begin{document}
\maketitle
\section*{Abstract}
This project was fun, rewarding, had awesome results and yielded in like fifty internships at google, each!
\newpage
\tableofcontents
\newpage

\section{Introduction}

We've chosen to do a version of the text analyser project, where we focus on categorizing either the content of the text or the type of text, depending on the difficulty level of the two alternatives. We will do this by implementing a heuristic that uses hidden Markov models, where we will research which variables should be included and give them a certain weight. If we have enough time, we will try to improve the heuristic by applying some sort of machine learning to the heuristic. If this is the case, it will probably be a simple version of policy gradient reinforcement learning (pgrl). We will work in python \textgreater= 3.0 and use some sort of natural language library to do the statistical analysis of the texts. Example libraries are TextBlob (https://textblob.readthedocs.org/en/dev) and the Natural Language Tool Kit (http://www.nltk.org)
\newline

\section{Method}  

\section{Results}

\section{Discussion}

\end{document}
